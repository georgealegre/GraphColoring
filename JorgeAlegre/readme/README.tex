\documentclass[11pt]{article}   	% use "amsart" instead of "article" for AMSLaTeX format
\usepackage{geometry}                		% See geometry.pdf to learn the layout options. There are lots.
\geometry{a4paper}                   		% ... or a4paper or a5paper or ... 
\usepackage{graphicx}				% Use pdf, png, jpg, or eps§ with pdflatex; use eps in DVI mode
								% TeX will automatically convert eps --> pdf in pdflatex		
\usepackage{amssymb}

\usepackage[spanish]{babel}

\usepackage[utf8]{inputenc}
\usepackage[T1]{fontenc}
\usepackage{lmodern} 				% load a font with all the characters

\usepackage{listings}
\lstdefinestyle{custombash}{
language=bash,
breaklines=true,
basicstyle=\ttfamily}

\lstdefinestyle{customC}{
language=C,
breaklines=true,
basicstyle=\ttfamily}

%SetFonts

%SetFonts


\title{\vspace*{\fill}Matemática Discreta II}
\author{Alegre, Jorge Facundo}
\date{\today}

\begin{document}

\topskip0pt
\maketitle
\thispagestyle{empty}
\begin{center}
Presentación - Proyecto, primera parte.
\end{center}
\vspace*{\fill}
\clearpage

\tableofcontents
\clearpage

\section{Introducción}
El siguiente documento detalla la implementación de un programa escrito en C para encontrar una cantidad
de colores que alcanzan para colorear de manera propia un grafo con el algoritmo Greedy de coloreo.
Vale aclarar que el programa \textbf{no indica cómo colorear el grafo}, solo indica cuantos colores se
pueden usar para colorearlo de forma propia.

En el documento encontrará una explicación de cómo compilar y ejecutar el programa, 
estructura elegida para el código, tipos de datos creados y decisiones de diseño
tomadas a lo largo de su desarrollo.
Además, puede encontrar respuestas a preguntas comunes que podrían surgir.
Por último, contendrá estadísticas de su comportamiento al ser ejecutado con distintos grafos de entrada.

El programa recibe un grafo en formato DIMACS (con algunas modificaciones TODO: agregar referencia). Después
de verificar que el formato de entrada se respete, el programa procede a verificar que $\chi(grafo)\neq2$. Si
es así, se crean 10 órdenes aleatorios para los vértices del grafo y se corre el algoritmo Greedy
sucesivamente. Luego, se ordena el grafo de forma decreciente con respecto al grado de cada vértice (orden
Welsh Powell) y se vuelve a correr Greedy. Si alguno de estos 11 órdenes colorea
el grafo con 3 colores (i.e., $\chi(grafo)=3$) la ejecución del programa se detiene.

En caso contrario, se vuelve a ordenar el grafo con el orden que produjo el mejor coloreo de las 11
iteraciones. Luego, comienzan
1001 iteraciones de ordenar y colorear el grafo. En estas iteraciones, el grafo se ordenará según sus
colores. Un teorema de coloreo de grafos prueba que al colorear un grafo con Greedy, si luego se lo ordena
según sus colores, volver a colorearlo con Greedy dará un coloreo con una cantidad de colores menor o igual
que el coloreo anterior. Los órdenes son Revierte, ChicoGrande, GrandeChico y ReordenAleatorioRestringido.
Serán explicados con mayor detalle. Estos órdenes son elegidos al azar, pero con distintas probabilidades
de ser elegidos, para las primeras 1000 iteraciones. La última siempre será Revierte.

Finalmente, se imprime por \texttt{STDOUT} el mejor coloreo obtenido y la cantidad de veces que se ordenó
con cada orden.
\clearpage

\section{Como compilar y usar}
Correr el siguiente comando, situados en el directorio \texttt{JorgeAlegre/}:
\lstset{style=custombash}
\begin{lstlisting}
$ gcc -Wall -Wextra -03 -std=c99 -Iapifiles 
   dirmain/mainJAlegre.c apifiles/*.c -o dirmain/color
\end{lstlisting}

Para correr el programa, simplemente lo ejecutamos de la siguiente forma:
\begin{lstlisting}
$ dirmain/color
\end{lstlisting}
En este momento, podemos ingresar los datos del grafo en el formato especificado. El programa
solo recibe datos por \texttt{STDIN}.

Si poseemos un grafo en un archivo, se lo puede cargar directamente de la siguiente forma:
\begin{lstlisting}
$ dirmain/color < archivo_de_grafo
\end{lstlisting}

Luego de cargar un grafo, el programa puede devolver un error ocasionado por un grafo de entrada
que no respeta el formato especificado o puede devolver el resultado de su corrida por \texttt{STDOUT}.
\clearpage

\section{Estructura del código}
\subsection{\texttt{apifiles/}}
La mayor parte del código se encuentra ubicado en el archivo \texttt{Cthulhu.c}. Esto es así porque
la mayoría de las funciones implementadas necesitan conocer la estructura interna del grafo, la cual esta
declarada al comienzo de este archivo.

\subsubsection{\texttt{Cthulhu.c}}
Posee la implementación de los tipos de datos \texttt{NimheSt} y \texttt{neighbours\_t}.
Contiene la implementación de todas las funciones declaradas en \texttt{Cthulhu.h}, las cuales serán usadas
en el \textbf{main()} del programa.

Además, posee la implementación de todas las funciones relacionadas
con el TAD \texttt{neighbours\_t}, las cuales estan detalladas en la sección \textbf{Tipos de datos}.

También se encuentra la implementación de todas las funciones de comparación usadas para ordenar el grafo.
Para ordenar el grafo, usaremos una función de C llamada \texttt{qsort()}, la cual es importada de
la librería \texttt{<stdlib.c>}. Ésta recibe, entre otras cosas, el arreglo a ordenar y la función usada
para comparar los elementos del arreglo.

\subsubsection{\texttt{Cthulhu.h}}
Posee la implementación del tipo de datos \texttt{VerticeSt}. Se encuentra en este archivo (i.e., su estructura es pública)
para que un tercero como el \textbf{main()} pueda llamar a la función \texttt{IesimoVerticeEnElOrden(NimheP G, u32 i)}, la cual devuelve un vértice.

\subsubsection{\texttt{helpers.c}}
Posee la implementación de las funciones usadas por la función \texttt{NuevoNimhe()} (i.e., las funciones
usadas para leer los datos del grafo por \texttt{STDIN}). Estas son:
\begin{itemize}
\item \texttt{readline\_from\_stdin()}

Devuelve un \emph{string} con la línea leída. En caso de no haber más líneas para leer o si una línea no
respeta el formato de entrada (e.g., línea tiene más de 80 caracteres)
devuelve \texttt{NULL}. Se la llama dentro de un ciclo para procesar lo que el usuario ingresa.
\item \texttt{handle\_p\_line()}

Si la línea empieza con la letra \textbf{p}, verificamos que sea de la forma \emph{p edge n m} donde n y m
son número de vértices y número de lados respectivamente, y devolvemos n y m encapsulados en una estructura
\texttt{par\_t}.
\item \texttt{handle\_e\_line()}

Si la línea empieza con la letra \textbf{e}, verificamos que sea de la forma \emph{e v w} donde \textbf{v} y \textbf{w} son los
nombres de los vértices que forman parte de la arista \textbf{vw}. Al igual que antes, devolvemos los datos
encapsulados en un \texttt{par\_t}. 
\item \texttt{strstrip()} encargada de remover el \texttt{\char`\\n} del final de la línea leída
\item \texttt{strcopy()} encargada de hacer copias de texto
\item \texttt{get\_l()} y \texttt{get\_r()}, para obtener ambos elementos de la estructura \texttt{par\_t}
\end{itemize}

\subsubsection{\texttt{rbtree.c}}
Posee la implementación de la API de los árboles Rojo-Negro, además de funciones auxiliares necesarias.
La especificación de esta API se encuentra en \texttt{rbtree.h}.
Las funciones públicas son:

\begin{itemize}
\item \texttt{rb\_new()}, encargada de pedir memoria para el árbol nuevo
\item \texttt{rb\_insert()}, para insertar nodos nuevos
\item \texttt{rb\_exists()}, para saber si un elemento existe
\item \texttt{rb\_search()}, para obtener el valor asociado a un elemento
\item \texttt{rb\_destroy()}, para destruir el árbol y liberar su memoria
\end{itemize}

\subsubsection{\texttt{u32queue.c}}
Posee la implementación de la API de la cola de números \textbf{u32}.
La especificación de esta API se encuentra en \texttt{u32queue.h}.
Las funciones públicas son:

\begin{itemize}
\item \texttt{queue\_empty()}, encargada de pedir memoria para la cola nueva 
\item \texttt{queue\_destroy()}, para destruir la cola y liberar su memoria
\item \texttt{queue\_is\_empty()}, para saber si hay elementos en la cola
\item \texttt{queue\_enqueue()}, para agregar elementos
\item \texttt{queue\_dequeue()}, para eliminar el primer elemento
\item \texttt{queue\_first()}, para leer los datos del primer elemento
\end{itemize}

\subsubsection{\texttt{typedefs.h}}
Dentro esta declarado del tipo de datos \textbf{u32}.

\subsection{\texttt{dirmain/}}
\subsubsection{\texttt{mainJAlegre.c}}
Dentro de este directorio existe un solo archivo, \texttt{mainJAlegre.c}. En él esta implementado el
\textbf{main()} de nuestro programa. 

Se encarga de:
\begin{itemize}
\item cargar el grafo y alertarle al usuario si se respeta el formato de entrada
\item contar cuántas veces se ordena con cada orden
\item verificar que $\chi(G)>3$ con la función \texttt{Greedy(G)}
\item destruir el grafo con la función \texttt{DestruirNimhe(G)}
\item generar los 10 órdenes aleatorios y obtener el coloreo con cada uno

Para esto, creamos un arreglo de tamaño \emph{n} con los valores del \emph{0} hasta \emph{$n-1$}.
Luego, recorremos el arreglo \textit{swappeando} la posición actual con una posición aleatoria.
Este arreglo se lo pasamos a la función \texttt{OrdenEspecifico(G, orden)}.

Siempre tendremos 2 arreglos, uno con el orden usado en el momento y otro con el orden que generó el mejor coloreo hasta el momento.
Solo reordenamos el del momento. Esto nos sirve para empezar las 1001 iteraciones con el mejor
orden usado en estas 10 iteraciones, a menos que el orden Welsh Powell dé mejor resultado.
Para generar números aleatorios, se llama a la función \texttt{rand()}.
\item ordenar el grafo con el orden Welsh Powell y obtener el coloreo
\item ordenar el grafo con los órdenes elegidos al azar y obtener el coloreo con cada uno

Un número aleatorio es generado y según su valor, ordenaremos con alguno de los 4 órdenes basados en los
colores del grafo.
\item imprimir en pantalla el resultado de las iteraciones
\end{itemize}
\clearpage

\section{Tipos de datos}
\subsection{u32}
El tipo de datos \textbf{u32} se usa alrededor de todo el programa. 
Para hacer su uso fácil, es definido en el archivo \texttt{apifiles/typedefs.h}
e incluído donde sea necesario. Es un alias al tipo de datos \texttt{uint32\_t}
incluído de la biblioteca estándar de C \texttt{<stdint.h>}.

\subsection{VerticeSt}
Está definido en el archivo \texttt{apifiles/Cthulhu.h}. Contiene información
básica de un vértice. Lo que \textbf{no} contiene es información de otros vértices,
por más que en el grafo estén relacionados.

Incluye el nombre real del vértice,
el cual es leído de lo que el usuario suministra al ejecutar el programa, su grado,
el cual se incrementa a medida que se agregan vecinos nuevos, y su color, usado
para ordenar el grafo. 

La estructura de este tipo es pública para que se puedan
crear o manipular vértices en el archivo \texttt{dirmain/mainJAlegre.c}.

\subsection{NimheSt - NimheP}
La estructura (NimheSt) está definida en \texttt{apifiles/Cthulhu.c} y un
puntero a la estructura (NimheP) en \texttt{apifiles/Cthulhu.h}. Se usa un
puntero para encapsular la información. Contiene los datos esenciales de un
grafo, como:
\begin{itemize}
\item la cantidad de lados
\item la cantidad de vértices
\item el $\Delta(grafo)$
\item un arreglo con todos los vértices cargados
\item un arreglo con la posición de los vértices en orden
\item un arreglo de la estructura de vecinos para cada uno de los vértices
\emph{(el elemento en la posición \textbf{i} de este arreglo corresponde con el vértice en
la posición \textbf{i} del arreglo de vértices)}
\item la cantidad de colores usados para colorearlo
\item un arreglo donde la posición indica el color y su valor, la cantidad de
vértices coloreados con ese color.
\end{itemize}

\subsection{neighbours\_t}
Es una estructura definida dentro de \texttt{apifiles/Cthulhu.c} ya que las
funciones que manipulan esta estructura necesitan conocer la estructura del
grafo. Su único objetivo es reunir dos arreglos en una sola estructura. Es usado
para representar a los vecinos de los vértices. Su relación con los vértices es
la posición en el arreglo (i.e., un vértice en la posición \textbf{i} tiene a sus vecinos
en el arreglo de vecinos en la misma posición).

El arreglo \texttt{neighbours}
contiene la posición de cada uno de los vértices vecinos del vértice en cuestión.

El arreglo \texttt{colors} es un \emph{bool vector} usado para colorear al vértice.
En este arreglo, la posición $i$ corresponde con el color $i+1$.

Las funciones implementadas para manipular este tipo son:
\begin{itemize}
\item \texttt{neighbours\_t neighbours\_empty()}

Crea la estructura usada por este tipo de datos. Devuelve un puntero a la estructura.

\item \texttt{neighbours\_t neighbours\_init(neighbours\_t neighbours, u32 size)}

Crea el arreglo \texttt{colors}. 
Recibe un puntero a la estructura y un \textbf{u32} indicando la cantidad de vecinos del vértice. Devuelve
un puntero a la estructura.

\item \texttt{neighbours\_t neighbours\_destroy(neighbours\_t neighbours)}

Libera toda la memoria usada por la estructura. Recibe un puntero a la estructura. Devuelve \texttt{NULL}.

\item \texttt{u32 neighbours\_i(neighbours\_t neighbours, u32 i)}

Recibe un valor entre 0 y el \emph{grado del vértice - 1}. Devuelve la posición del
iésimo vecino del vértice.

\item \texttt{neighbours\_t neighbours\_append(neighbours\_t neighbours, u32 index)}

Agrega la posición de un vecino nuevo del vértice. 
Recibe un puntero a la estructura y un \textbf{u32} indicando la posición de un vecino nuevo del vértice.
Devuelve un puntero a la estrucutra.

\item \texttt{u32 neighbours\_find\_hole(neighbours\_t neighbours, u32 grado)}

Recibe un puntero a la estructura. Devuelve el color más chico no usado por los vecinos de un vértice. Se supone
que se llama a \texttt{neighbours\_update()} antes de correr esta función.
Recorre el arreglo \texttt{colors} y devuelve la posición del primer
\texttt{false} $+1$.

\item \texttt{neighbours\_t neighbours\_update(NimheP G, u32 vertex)}

Actualiza el \emph{bool vector} \texttt{colors} marcando los colores usados por
los vecinos del vértice, es decir marca con \texttt{true} a la celda en la
posición \emph{color del vecino - 1}. Solo tiene en cuenta colores menores al
grado del vértice y distintos de 0 (vértices coloreados), el resto no importan.
\end{itemize}
\clearpage

\section{Decisiones de diseño}
\subsection{\texttt{rand()} y \texttt{srand()}}
Para obtener números aleatorios, se llama a la función \texttt{rand()} de la librería estandar de C. Como semilla se usa
$(\#lados(grafo) * \#vértices(grafo)) + nombre(2ºvértice)$.

\subsection{\texttt{realloc()} para agregar vecinos}
Para agregar los vecinos de un vértice, llamamos a la función \texttt{neighbours\_append()}. Al no saber la cantidad de vecinos que tendrá
un vértice a la hora de crearlo, no se puede saber el tamaño del arreglo de los vecinos del vértice, entonces se usa un factor constante para ir
agrandando su tamaño en \emph{runtime}, si al agregar un vecino nuevo, el arreglo ya tiene todas sus posiciones ocupadas. Esta constante, definida en
\texttt{apifiles/Cthulhu.c} vale 8. Esto desperdicia memoria pero mejora el rendimiento del programa. A medida que este factor crece, se gana velocidad
pero se pierde eficiencia espacial.
\clearpage

\section{Respuestas a preguntas particulares}
\subsection{Dado que los vértices pueden ser cualquier u32, ¿cómo sabemos si un vértice recién leído ya
lo habíamos leído antes?}
Al leer los vértices suministrados
por el usuario, queremos saber si es un vértice nuevo para alocar espacio para contenerlo o si simplemente
es un vértice existente y se quiere aclarar que tiene como vecino al vértice que lo acompaña.
Dado que los vértices pueden ser cualquier número entre 0 y $2^{32}-1$, resolver este problema no es trivial
si se desea velocidad alta y uso de memoria bajo para cargar el grafo.

El método usado en esta implementación es usar árboles Rojo-Negro (TODO: referencia). Con esta estructura
de datos, podemos lograr una eficiencia de espacio de $\mathcal{O}(n)$, siendo \emph{n} la cantidad de
vértices en el grafo. Las operaciones sobre el árbol usadas por el programa son las de insertar elementos
nuevos y buscar elementos, ambas con complejidad de tiempo de $\mathcal{O}(\ln{n})$.

\subsection{¿Cómo esta implementado $\Gamma(x)$?}
El grafo, en su estructura interna, posee un arreglo de los vértices y un arreglo de arreglos de los
índices (en el arreglo de los vértices) de cada uno de los vecinos de los vértices.

Por ejemplo, si en el índice \emph{3} del arreglo de los vértices se ubica el vértice \emph{35}, en el
índice \emph{3} del arreglo de los vecinos se ubica una estructura del tipo \textbf{neighbours\_t}, el cual
encapsula información de los vecinos del vértice \emph{35}, incluido un arreglo con los índices de los
vértices vecinos del vértice \emph{35}. Esta información se obtiene mediante la función 
\texttt{neighbours\_i()}.

\lstset{style=customC}
\subsection{¿Cómo esta implementado el orden de los vértices?}
Dentro de la estructura del grafo existe un arreglo de números de tipo \textbf{u32} de tamaño \emph{n}.
Este arreglo, en cada posición, contiene los índices de los vértices en el arreglo de vértices.
Las posiciones indican el orden, es decir, supongamos que el arreglo con el orden se llama \texttt{orden}
y el arreglo de vértices se llama \texttt{vertices}. Entonces,
\begin{lstlisting}
vertices[orden[0]]
\end{lstlisting}
equivale al primer vértice del grafo en el orden actual. Por otro lado,
\begin{lstlisting}
vertices[0]
\end{lstlisting}
equivale al primer vértice del grafo en el orden que el programa leyó los vértices.

El arreglo con los vértices nunca intercambia de lugar sus elementos. El arreglo con el orden es modificado
con todas las funciones para ordenar el grafo.

Desde que se carga un grafo hasta que el programa termina, el primer vértice en el arreglo de vértices es
siempre el mismo. El valor en el primer campo del arreglo con el orden no siempre será igual.

\subsection{¿Cómo esta implementada la función \texttt{Greedy()}?}
Primero, le quita el color a todos los vértices. Esto es necesario para que \texttt{neighbours\_update()}
no tenga en cuenta colores de corridas viejas. También limpiamos el arreglo \texttt{nvertices\_color},
dejando todos sus campos con el número 0.

Luego, coloreamos el primer vértice con el color 1.

Después, en un ciclo de \emph{1...n} (el primer vértice ya está coloreado) obtenemos la posición de los
vértices en orden. Actualizamos su estructura de vecinos asociada, es decir, recorremos todos los vecinos
del vértice en cuestión, viendo sus colores y ajustando el arreglo \texttt{colors}. Este arreglo tiene
una cantidad de celdas igual al grado del vértice. Esto es porque si un vértice tiene 24 vecinos, y uno de
esos vecinos está coloreado con el color 28, no afecta en el color que seleccionaré para colorear al vértice.
En el peor caso, todos los vecinos del vértice estan coloreados con colores distintos, crecientes a partir
del color 1. En el ejemplo, supongamos que la posición del vértice en cuestión es 5,
\lstset{style=customC}
\begin{lstlisting}
G->vertices[5].grado = 24
G->vertices[neighbours_i(G->vecinos[5], 0)].color = 1
G->vertices[neighbours_i(G->vecinos[5], 1)].color = 2
.
.
.
G->vertices[neighbours_i(G->vecinos[5], 23)].color = 24
\end{lstlisting}
Entonces, luego de actualizar el arreglo de colores con la función \texttt{neighbours\_update()}, el arreglo
de colores nos queda así:
\begin{lstlisting}
colors[0] = true
colors[1] = true
.
.
.
colors[23] = true
\end{lstlisting}
La función \texttt{neighbours\_find\_hole()}, que devuelve el color más chico no usado, recorre el arreglo
de colores buscando alguna posición con el valor \texttt{false}, indicando que el color esta libre para
usar. En el ejemplo, ésta función devolvería 25. 

Con el color que devuelve la función coloreamos al vértice y aumentamos por 1 la cantidad de vértices
coloreados con ese color.

Al salir del ciclo, lo único pendiente por hacer es contar cuántos colores fueron usados para colorear
el grafo. Para esto, recorremos el arreglo \texttt{nvertices\_color} hasta que el valor en algúna celda
sea 0.

Esta función podría ser más rapida ya que para cada vértice, estamos recorriendo todos sus vecinos, no solo
los vecinos que fueron coloreados antes de él. No produce un coloreo no propio porque limpiamos los colores
del grafo antes de arrancar. Con la implementación elegida para el grafo, tener en cuenta los vértices ya
coloreados y revisar solo los vecinos de cada vértice ya coloreados costaría más que hacerlo de esta manera.
\subsection{¿Cómo estan implementados los órdenes Revierte, ChicoGrande y GrandeChico?}
Estos 3 órdenes están implementados de la misma manera, usando la función \texttt{qsort()}. Lo único que
tienen de distinto es que la función de comparación usada para cada uno es diferente.

\end{document}  
